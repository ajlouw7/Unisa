\documentclass[10pt,a4paper]{article}
\usepackage[natbibapa]{apacite} 
\usepackage{amsmath}
\usepackage{tikz}
\bibliographystyle{apacite}

\title{Assignment 2 Machine Learning COS4852}
\author{ Adriaan Louw (53031377) }

\tikzset{
  treenode/.style = {shape=rectangle, rounded corners,
                     draw, align=center,
                     top color=white, bottom color=blue!20},
  root/.style     = {treenode, font=\Large, bottom color=red!30},
  env/.style      = {treenode, font=\ttfamily\normalsize},
  dummy/.style    = {circle,draw}
}

\usetikzlibrary{shapes.misc}

\tikzset{cross/.style={cross out, draw=black, minimum size=20*(#1-\pgflinewidth), inner sep=0pt, outer sep=0pt},
%default radius will be 1pt. 
cross/.default={1pt}}
\begin{document}

\maketitle

\tableofcontents

\section{Question 1}
\subsection{Question 1(a)}

Firstly we calculate the line 

\begin{equation}
\label{line}
 x_2 = mx_1 + c
\end{equation}

for the intersect points (2,0) and (0,6).

Calculating slope m,

\begin{equation}
\begin{split}
m &= \frac{6-0}{0-2} \\
  &= -3\\
\end{split}
\end{equation}

$x_2$ intercept $c$ is 6.

This makes equation \ref{line}

\begin{equation}
\label{pop}
x_2 = -3x_1 + 6
\end{equation}

\cite{nils} gives the equation for the hyperplane as

\begin{equation}
\sum_{i=1}^n x_i\omega_i \geq \theta
\end{equation}

which in this case gives the equation for the hyperplane to be

\begin{equation}
\label{weightline}
\omega_1x_1 + \omega_2x_2 +\omega_3 = 0
\end{equation}

We need to get equation \ref{weightline} in the form of equation \ref{line}

\begin{equation}
\begin{split}
\label{simp}
\omega_1x_1 + \omega_2x_2 + +\omega_3 &= 0\\
\omega_2x_2 &= -\omega_1x_1 - \omega_3\\ 
x_2 &= \frac{\omega_1x_1}{\omega_2} - \frac{\omega_3}{\omega_2} \\ 
\end{split}
\end{equation}

Comparing coefficients m and c from equation \ref{pop} to \ref{simp} we get

\begin{equation}
\begin{split}
-\frac{\omega_1}{\omega_2} &= -3 \\
\omega_1 &= 3\omega_2\\ 
\end{split}
\end{equation}

and

\begin{equation}
\begin{split}
-\frac{\omega_3}{\omega_2} &= 6\\
\omega_3 &= -6\omega_2\\ 
\end{split}
\end{equation}

If we choose $\omega_3 = -2$ then $\omega_1=1$ and $\omega_2 = \frac{1}{3}$. This makes the hyperplane equation from equation \ref{weightline}

\begin{equation}
x_1 + \frac{x_2}{3} -2 = 0
\end{equation}

Now we need to test this hyperplane. For positive instance (2,6)

\begin{equation}
\begin{split}
x_1 + \frac{x_2}{3} - 2 &= \\
2 + \frac{6}{2} - 2 &= \\
2&\\
\end{split}
\end{equation}

Which is as expected.

And the negative instance (-1,2)

\begin{equation}
\begin{split}
x_1 + \frac{x_2}{3} - 2 &= \\
-1 + \frac{2}{3} -2 & = \\
-\frac{7}{3}
\end{split}
\end{equation}

This is also as expected. The perceptron now classifies the the data correctly
\subsection{Question 1(b)}

\begin{center}
\begin{tikzpicture}[scale=0.6]
%\draw [help lines] (-3,-3) grid (3,3);
% Euclidean
\draw [<->](0,-8)--(0,8) node[right]{$X_2$};
\draw [<->](-8,0)--(8,0) node[right]{$X_1$};
 
%labels
\foreach \x in {-8,-6,-4,-2,2,4,6,8}
     \draw (\x,1pt) -- (\x,-3pt) node[anchor=north] {$\x$};

\foreach \y/\ytext in {-8,-6,-4,-2,2,4,6,8}
     \draw (1pt,\y) -- (-3pt,\y) node[anchor=east] {$\y$};
 
\draw (-1,2) node[cross,green]{};
\draw (1,-2) node[cross,green]{};
\draw (-2,-5) node[cross,green]{};
\draw (-5,-2) node[cross,green]{}; 

\draw (-5,6) node[cross,red]{};
\draw (3,5) node[cross,red]{};
\draw (2,-2) node[cross,red]{};
\draw (5,-5) node[cross,red]{};

\draw [blue] (-8,8) -- (8,-8);
\draw [teal] (-8,66/7) -- (8,-62/7);
 
\end{tikzpicture}
\end{center}

From the above image we can see that any


\subsection{Question 1(c)}
\subsection{Question 1(d)}

\section{Question 2}
\subsection{Question 2(a)}
\subsection{Question 2(b)}
\subsection{Question 2(c)}

\section{Question 3}
\subsection{Question 3(a)}
\subsection{Question 3(b)}
\subsection{Question 3(c)}
\subsection{Question 3(d)}
\subsection{Question 3(e)}

\bibliography{mybib}
\end{document}
