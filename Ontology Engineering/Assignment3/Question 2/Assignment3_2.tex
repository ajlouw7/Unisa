\documentclass[12pt,a4paper]{article}
\usepackage[natbibapa]{apacite}
\usepackage[a4paper,margin=1.3in]{geometry} 
\bibliographystyle{apacite}

\title{Ontology Engineering Assignment 3}
\author{Adriaan Louw (53031377)}

\begin{document}
\section{Question 2}

\subsection{Nature and Purpose of Foundational Ontologies}

According to \cite{dolce}, foundational ontologies have the following characteristics: "(i) have a large scope, (ii) can be highly reusable in different modeling scenarios, (iii) are philosophically and conceptually well founded, and (iv) are semantically transparent and (therefore) richly axiomatized."

Numerous foundational or upper level ontologies exist. For example
SUMO \citep{sumo},
DOLCE \citep{dolce},
BFO \citep{BFO},
BORO \citep{DeCesare},
and UFO \citep{DeCesare}.

Some believe that there will eventually be only a couple of foundational ontologies. Where people will prefer to share ontologies rather than have to translate between them constantly \citep{sumo}.

Advantages of using an ontology
1. reuse 
Using a foundational ontology lets the ontology designer use concept and models that have been thoroughly tested and designed by experts in the field \citep{sumo}. This speeds up development of the ontology, hence reduces cost and reduces errors. This can be put another way. Foundational ontologies provide "general architectural infrastructure for roles" \citep{DeCesare}.

2. Using a foundational ontology can reduce the amount of rework that needs to be done on an ontology due to changing requirements. Such ontologies anticipate changes because they have been designed to handle challenging modelling situations \citep{sumo}.



\subsection{SUMO}
\citep{sumo}

Numerous foundational ontologies were combined to create the Suggested Upper Merged Ontology (SUMO)\citep{sumo}. According \cite{sumo}, amongst those are ontologies from ITBM-CNR, Stanford KSL and content based on the works of Sowa \citep{sowa}, Guarino and colleagues \citep{borgo1996pointless}, Allen \citep{J.F.Allen1984} and Smith \citep{Smith1996}.

There are 11 sections with documented interdependencies. These include sections on structure of relation, entity abstraction, graph theory, and units of measure \citep{sumo}.



\subsection{DOLCE}
\citep{dolce}




endurantism vs perdurantism need to accomodate both

\subsection{BFO}

\bibliography{mybib}
\end{document}