\documentclass[12pt,a4paper]{article}
\usepackage[natbibapa]{apacite}
\usepackage[a4paper,margin=1.3in]{geometry} 
\bibliographystyle{apacite}

\title{Ontology Engineering Assignment 3}
\author{Adriaan Louw (53031377)}

\begin{document}
\maketitle
\newpage
\section{Question 1}

Snomed CT (Systematized Nomenclature of Medicine Clinical Terms) is a ".. the most comprehensive clinical terminology system" \citep[p1]{analysis}. It uses the formal top-level ontology DOLCE (Descriptive Ontology for Linguistic and Cognitive Engineering) as its base \citep[p2]{analysis}.  

\subsection{History}

SNOMED CT can be traced back to 1965 when the College of American Pathologists published the Systematized Nomenclature of Pathology (SNOP). SNOP covered a large number of known medical conditions, since it was designed for the classification of surgical and autopsy diagnoses \citep{Cote1980}.

This was further expanded into SNOMED in 1977. Unlike SNOP, SNOMED was a collaboration between various societies and not just the College of American Pathologists. It included various medical societies from the United States, England and Canada \citep{Cote1980}. 

SNOMED RT was a new logic based version published in 2000, in colaboration with Kaiser Permanente \citep{sn2018,nih}. SNOMED RT was merged with Clinical Terms Version 3 (CTV3) and released as SNOMED CT in January 2002. \citep{sn2018}

In 2007 the International Health Terminology Standards Development Organisation gained the intellectual rights to all SNOMED versions \citep{nih}.

\subsection{Purpose \& Use}

According to \cite{snomedcttech2015} there are 3 types of implementations for SNOMED CT:

\begin{itemize}
\item \emph{Clinical Records}: These implementations manage records of patients. Such systems can vary insize from specialized departments to large organizations. These systems provide means of entering new SNOMED CT expressions, storing and retrieving them. \citep{snomedcttech2015}

\item \emph{Knowledge Representation}: SNOMED CT is used in knowledge resources like decision support protocols, electronic reference books and clinical guidelines. In these implementations, SNOMED CT expressions are used to define concepts and relationships. In more sophisticated uses queries can be generated on stored information, like in a clinical decision support system.\citep{snomedcttech2015}

\item \emph{Aggregation and analysis}: The process of gathering data for later analysis.\citep{snomedcttech2015}
\end{itemize}

Using an ontology like SNOMED CT comes with a lot of benefits. It creates a " ...consistent way of indexing, storing, retrieving and aggregating clinical data across specialties and sites of care" \cite[p2]{y40}.

Data can be easily reused for other purposes. For example if the organisation also use ICD codes, SNOMED CT can generate these codes based on the diagnosis entered in the SNOMED CT enabled system\cite[p92]{Lee2013}.

A SNOMED CT Query language is being developed by SNOMED International. It does not build On W3C Web Ontology Language(OWL). But it does allow for SNOMED CT to be transformed into OWL using a perl script \citep[p218]{analysis}.

\subsection{Ongoing Maintenance}

There have been a number of issues identified in implementations of SNOMED CT. Certain terms have been found to be ambiguous \cite[p92]{Lee2013}. Like the term for "cold", does it refer to the "common cold" ("82272006$\mid$Common cold (disorder)$\mid$") or a "cold injury" ("11925005$\mid$Effects of reduced temperature (disorder)$\mid$")? Issues with hierarchical relationships have also been found. No subsumption relationship exists between the concepts "69973000$\mid$Vascular anomaly of eyelid (disorder)$\mid$" and "193966008$\mid$Eyelid vascular anomalies (disorder)$\mid$". The latter is only a congenital occurrence. even though both are conditions of the eye lid \cite[p92]{Lee2013}. Additionally issues were found where the use of hyphens, full stops and commas we used inconsistently \cite[p92]{Lee2013}.

There have been calls to IHTSDO for more guidance with respect to how to create subsets of SNOMED CT where the domain is large. For example reason for admittance \cite[p92]{Lee2013} A subset is a meaningful fragment of the ontology that encompasses everything related to a specific domain. This makes it easier for implementers an users to use. This fragment needs to be as small as possible but still contain all concepts that are needed for the specific domain \citep[p25]{subset}.

Large ontologies like SNOMED CT, that is in constant use, requires constant maintenance hence a new version of the ontology is published every 6 months \citep{snocr}. This leads to an an additional challenge for clinicians. After every update the hierarchy has changed therefore the queries could also need to change over time \cite[p92]{Lee2013}.

\subsection{Conclusion}
SNOMED CT is officially used in over 50 countries. But there are not a lot of implementations. It still has some issues to work out. But it has been of benefit to some \cite[p93]{Lee2013}. 

\bibliography{mybib}
\end{document}