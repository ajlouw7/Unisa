\documentclass[12pt,a4paper]{article}
\usepackage[natbibapa]{apacite} 
\bibliographystyle{apacite}

\begin{document}

\textbf{Question 2}
\newline

Today's content on the web is mostly human readable and not machine readable. This it difficult for machine to independently, form a human intelligence, find ind deduce information. Current systems can only to text searches but cannot deduce meaning. Hence the development of the Semantic web. "The Semantic Web will bring structure to the meaningful content of Web pages, creating an environment where software agents roaming from page to page can readily carry out sophisticated tasks for users". \citep{lee2001} Examples of services the Scemantic Web will enable are information brokers, search agents and information filters. \citep{Decker} 

Agents will be able to deduce the meaning of the data and infer new knowledge from what it already knows. Because the agents will be using the semantics embedded in the web pages, we will not need AI with a level of sophistication of a human being. \citep{lee2001}

The semantic information has to be encoded into webpages using a machine readable language for these agents to function. With this in mind, there are a couple of languages that have come forward. RDF and RDF SCHEMA were the initial focus of the but we found to be lacking in expressive power. The World Wide Web Consortium (W3C) created the Web Ontology Group to develop an Ontology language for the Semantic Web. The relationship between terms are to be provided by this ontology in a structured vocabulary. Making it easy for agent to interpret unambiguously \citep{Horrocks2003}

Using common languages allows agents to search through different webpages to find relevant data. But these languages do not guarantee that the meaning of particular terms are the same across different we pages or domains. 



\bibliography{mybib}
\end{document}
