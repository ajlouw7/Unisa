\documentclass[10pt,a4paper]{article}
\usepackage{amsmath}
\usepackage{textcomp}
\usepackage[a4paper,margin=0.8in]{geometry} 
\usepackage{natbib}
\bibliographystyle{agsm}

%opening
\title{COS 4807 Assignment 1}
\author{Adriaan Louw (53031377)}

\begin{document}

\maketitle

\tableofcontents

\listoffigures

\listoftables


\section{Abstract}
hello
\section{Introduction}

Human attempts to mathematically predict life expectance is not a new endeavour. \cite{Gompertz} introduced an equation to predict life expectancy, which was modified in \cite{Makeham1860} to create the famous Gompertz$-$Makeham law.


Why use machine learning? find relationships that regression analysis cannot \cite{Chen2017}.

Machine learning is used in medicine \cite{Chen2017}.

life expectancy vs mortality rate?

\cite{Rajkomar2018} Google uses machine learning to predict in hospital medical events for patients. 

\section{Literature Review}

Forecasting Mortality in Developed Countries Tabeau 2001

\subsection{?Grossman?}
2017 determinants of health: an economic perspective  ????
1972 The Demand for Health: A Theoretical and Empirical Investigation,

\cite{Grossman2000}

\subsection{Life expectancy projections}

The United Nations use a Bayesian model to predict future life expectancy \citep{Raftery2014}.

Lee Carter method \cite{Shang2011} later extended into the Li-Lee model

Siminal work \cite{Lee1992}

\cite{Bongaarts2005}

\subsection{Determinants of life expectancy}



\subsubsection{Income}

The relationship beteen income and life expectancy has been given a lot of attention in academic circles \citep{Preston1975, Hu2015, Chetty2016, Oeppen2019}. 











\cite{Kalwij2014}

\cite{Oeppen2019} Very Good!!

\cite{Preston1975} is a seminal work according to \cite{Oeppen2019}

inequality \cite{Hu2015}

\cite{Chetty2016} in the US

income inequality does not affect health of a a country \cite{JasonBeckfield2004}

unemployment \cite{Bonamore2015} \cite{Roelfs2011} \cite{Roelfs2015} 

\cite{Tarkiainen2012} (To be downloaded)

\subsubsection{Education atainment}


\cite{Kaplan2015} investigated the relationship between educational atainment and life expectancy in eight states in the United States. They found that even when controlling for variables like income, race, sex and common medical issues like cardiovascular disease, the relationship between educational antainment and life expectance remains statistically significant.

But what is the nature of this correlation? According to \cite{Deary2004} Inelligence Quotient or IQ could explain the association. While \cite{Hayward2015} does not believe in a ``causal relationship'' but rather that it depends on factors like ``time, place, and the social environment''.


Study in Belgium \cite{Deboosere2009}


Inverse relationship  \cite{Hoque2019}

netherlands \cite{VanKippersluis2009}

\cite{VanBaal2016a}

\subsubsection{Per capita spending on health}

\cite{Shaw2005} showed that pharmaceutical expenditures shows a positive correlation with life expectancy in OECD countries.

medical spending \cite{Cutler2006}

\subsubsection{Access to safe drinking water}

\subsubsection{Infant mortality}

\cite{CDC1999}

\subsubsection{Turmoil}
\citep{Low2008} p211

\subsection{The gender gap}

\cite{Rochelle2015}

\section{Methodology/Procedure}

There are many studies that attempt to extrapolate future life expectancy for countries based on current data. This includes studies for high income countries \citep{Kontis2017} and low income countries ????{cite}.

This study will attempt to create a model that can predict life expectancy for a country based on various socio-economic conditions in the country.



segment data into groups where each group has the same amount of data points???

Unlike \cite{Shaw2005}, this study will not take into account the age distribution of eacmh country.

As for HDI from \cite{Bulled2010}
Adult literacy rate

primary secondary and tertiary enrolment ratios

GDP per Capita (Purchasing power parity )

\subsection{Choice of dataset}

\subsection{Regression}

\subsection{k-Nearest Neighbour}

\subsection{Support Vector Machines}

\subsection{Cross-validation}

\section{Analysis}

\section{Conclusion}

\section{Recommendations}

\addcontentsline{toc}{section}{References}
\bibliography{mybib}

\section{Appendices}



\end{document}
