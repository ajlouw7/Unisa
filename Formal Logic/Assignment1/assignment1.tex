\documentclass[10pt,a4paper]{article}
\usepackage[natbibapa]{apacite} 
\usepackage{amsmath}
\usepackage{textcomp}
\usepackage[a4paper,margin=0.8in]{geometry} 

%opening
\title{COS 4807 Assignment 1}
\author{Adriaan Louw (53031377)}

\newcommand{\interp}{I}
\begin{document}

\maketitle

\section{Question 1.1}

For Formula


\begin{equation}
 ((( p \vee q) \vee r ) \wedge ( \neg p \wedge \neg q )) \rightarrow r
\end{equation}

To prove the formula is valid we need to show it is true for all possible interpretations. 
We will use 0 to represent false and 1 to represent true

\begin{tabular}{|c|c|c|c|c|c|c|c|c|c|}
\hline
$p$ & $q$ & $r$ & $\neg p$ & $ \neg q$ & $p \vee q$ & $(p \vee q) \vee r$ & $\neg p \wedge \neg q$ &  $((( p \vee q) \vee r ) \wedge ( \neg p \wedge \neg q ))$ & $ ((( p \vee q) \vee r ) \wedge ( \neg p \wedge \neg q )) \rightarrow r$\\ 
\hline
 0 & 0 & 0 & 1 & 1 & 0 & 0 & 1 & 0 & 1\\
 0 & 0 & 1 & 1 & 1 & 0 & 1 & 1 & 1 & 1\\
 0 & 1 & 0 & 1 & 0 & 1 & 1 & 0 & 0 & 1\\
 0 & 1 & 1 & 1 & 0 & 1 & 1 & 0 & 0 & 1\\
 1 & 0 & 0 & 0 & 1 & 1 & 1 & 0 & 0 & 1\\
 1 & 0 & 1 & 0 & 1 & 1 & 1 & 0 & 0 & 1\\
 1 & 1 & 0 & 0 & 0 & 1 & 1 & 0 & 0 & 1\\
 1 & 1 & 1 & 0 & 0 & 1 & 1 & 0 & 0 & 1\\
\hline
\end{tabular}

Since this formula is true in all cases (last column), it is valid (tautology).

\section{Question1.2}

For Formula

\begin{equation}
(p \rightarrow ( q \rightarrow r )) \leftrightarrow (( p \wedge q) \wedge \neg r) 
\end{equation}

to show that it is unsatisfiable we need to show that it is false for all interpretations

\begin{tabular}{|c|c|c|c|c|c|c|c|c|}
\hline
$p$ & $q$ & $r$ & $q \rightarrow r$ & $p \rightarrow ( q \rightarrow r )$ & $p \wedge q$ & $\neg r$ & $( p \wedge q) \wedge \neg r $ & $(p \rightarrow ( q \rightarrow r )) \leftrightarrow (( p \wedge q) \wedge \neg r) $\\
\hline
 0 & 0 & 0 & 1 & 1 & 0 & 1 & 0 & 0\\
 0 & 0 & 1 & 1 & 1 & 0 & 0 & 0 & 0\\
 0 & 1 & 0 & 0 & 1 & 0 & 1 & 0 & 0\\
 0 & 1 & 1 & 1 & 1 & 0 & 0 & 0 & 0\\
 1 & 0 & 0 & 1 & 1 & 0 & 1 & 0 & 0\\
 1 & 0 & 1 & 1 & 1 & 0 & 0 & 0 & 0\\
 1 & 1 & 0 & 0 & 0 & 1 & 1 & 1 & 0\\
 1 & 1 & 1 & 0 & 1 & 1 & 0 & 0 & 0\\
\hline
\end{tabular}

From this we can see that the Formula is unsatisfiable for all interpretations of p q and r

\section{Question1.3}

For Formula
\begin{equation}
(p \rightarrow (q \rightarrow r)) \rightarrow ((p \rightarrow q)  \rightarrow r)
\end{equation}

we need to show that there exists at least 1 interpretation for which this formula is false
and at least one interpretation for which this formula is true

\begin{tabular}{|c|c|c|c|c|c|c|c|}
\hline
$p$ & $q$ & $r$ & $q \rightarrow r$ & $p \rightarrow q$ & $p \rightarrow (q \rightarrow r)$ &$ (p \rightarrow q)  \rightarrow r$ & $(p \rightarrow (q \rightarrow r)) \rightarrow ((p \rightarrow q)  \rightarrow r)$ \\ 
\hline
 0 & 0 & 0 & 1 & 1 & 1 & 0 & 0\\
 0 & 0 & 1 & 1 & 1 & 1 & 1 & 1\\
 0 & 1 & 0 & 0 & 1 & 1 & 0 & 0\\
 0 & 1 & 1 & 1 & 1 & 1 & 1 & 1\\
 1 & 0 & 0 & 1 & 0 & 1 & 1 & 1\\
 1 & 0 & 1 & 1 & 0 & 1 & 1 & 1\\
 1 & 1 & 0 & 0 & 1 & 0 & 0 & 1\\
 1 & 1 & 1 & 1 & 1 & 1 & 1 & 1\\
\hline
\end{tabular}

The above trusth table shows that the formula is true for some interpretations and thus the Formula is satisfiable. 
Also the truth table shows that the formula is false for some interpretations and thus the formula is falsifiable















\section{Question2.1}

Given Formula 


\newcommand{\ti}{\nu_\interp}


\begin{equation}
 F_1  = ((( p \vee q) \vee r ) \wedge ( \neg p \wedge \neg q )) \rightarrow r
\end{equation}

We need to show that 
\begin{equation}
\label{21prove}
 \ti (F_1) = T
\end{equation}

for any interpretation of p, q and r.

Using proof by contradiction we assume

\begin{equation}
\label{21ass}
 \ti (F_1 ) = F
\end{equation}

then 

\begin{equation}
 \ti(r) = F
\end{equation}

 and
 
 \begin{equation}
 \label{21}
\ti((( p \vee q) \vee r ) \wedge ( \neg p \wedge \neg q )) = T  
 \end{equation}

The subformula 

\begin{equation}
(( p \vee q) \vee r ) \wedge ( \neg p \wedge \neg q )
\end{equation}

is made up of the conjuction between sub formulas

\begin{equation}
\label{211}
( p \vee q) \vee r 
\end{equation}

and 

\begin{equation}
\label{212}
\neg p \wedge \neg q 
\end{equation}

Thus the statement \ref{21} can only be valid if both the statements

\begin{equation}
 \ti(( p \vee q) \vee r ) = T
\end{equation}

and 

\begin{equation}
 \ti( \neg p \wedge \neg q ) = T
\end{equation}

are true.

But formula \ref{212} is the negation of formula \ref{211} by de Morgans law, taking into account that $\ti(r) = F$ and the disjunction of any statement with a false statement has the tuth value of the original statement. 
Thereore we have a contradiction. Statement \ref{21} cannot be true because the conjunction of a formula and its negation cannot be True for any interpretation. 
That means our assumption in statement \ref{21ass} is incorect and therefore statement \ref{21prove} is true. And since $F_1$ is true for all interpretations, it is a valid formula.

\section{Question 2.2}

Given formula

\begin{equation}
\label{220}
(p \rightarrow ( q \rightarrow r )) \leftrightarrow (( p \wedge q) \wedge \neg r) 
\end{equation}

We need to prove that 

\begin{equation}
\ti((p \rightarrow ( q \rightarrow r )) \leftrightarrow (( p \wedge q) \wedge \neg r) ) = F 
\end{equation}

Using proof by contradiction, we assume 

\begin{equation}
\ti((p \rightarrow ( q \rightarrow r )) \leftrightarrow (( p \wedge q) \wedge \neg r) ) = T
\end{equation}

We know that the equivalence operator only returns true when both sides of the forluma is true. In this case when

\begin{equation}
\label{221}
 \ti(p \rightarrow ( q \rightarrow r )) = F
\end{equation}

and 

\begin{equation}
\label{222}
 \ti(( p \wedge q) \wedge \neg r ) = F
\end{equation}


which we will call Case 1 and

\begin{equation}
\label{223}
\ti(p \rightarrow ( q \rightarrow r )) = T
\end{equation}

and 

\begin{equation}
\label{224}
 \ti(( p \wedge q) \wedge \neg r ) = T
\end{equation}


which we will call case 2.

For Case 1, formula \ref{221} can only be valid if 

\begin{equation}
 \ti(p) = T
\end{equation}

and

\begin{equation}
\label{2211}
 \ti( q\rightarrow r ) = F
\end{equation}

are valid. In turn, for eaqation \ref{2211} to be valid we have to have 

\begin{equation}
 \ti(q) = T
\end{equation}

and

\begin{equation}
 \ti(r) = F
\end{equation}

Thus in case 1 for equation \ref{221} to be valid we have to have the values $\ti(p) = T$, $\ti(q) = T$ and $\ti(r) = F$.
Now to prove Case 1 we have to use these values for interpretation \ref{222}. But that gives

\begin{equation}
 \ti((p\wedge)\wedge \neg r) = T
\end{equation}

which is a contradiction to equation \ref{222}. Which means formula cannot be true when wh have equations \ref{221} and \ref{222}.

Now for case 2:

Equation \ref{224} can only be true if $\ti(p) = T$, $\ti(q) = T$ and $\ti(r) = F$ due to the definition of conjuctions. 
Equation \ref{223} need to be true for this interpretation. But using these interpretations for p,q and r gives

\begin{equation}
\ti(p \rightarrow ( q \rightarrow r )) = F
\end{equation}

Which is a contradiction with equation \ref{223}.

In both cases when formula \ref{220} could have been True, we have a contradiction. 
The formula has to be false for all interpretations and is thus unsatisfiable.





\section{Question 3}










\section{Question 4.1}
For formula

\begin{equation}
\label{41}
 ((( p \vee q) \vee r ) \wedge ( \neg p \wedge \neg q )) \rightarrow r
\end{equation}

Applying the $\beta$ formula for implication gives

\begin{equation}
\label{411}
 \neg  ((( p \vee q) \vee r ) \wedge ( \neg p \wedge \neg q ))
\end{equation}

and

\begin{equation}
\label{412}
 r
\end{equation}

Where Formula \ref{412} is satisfiable.

Continueing for Formula \ref{411} by applying the $\beta$ formula for $\neg (B_1 \wedge B_2)$ we get



 \begin{equation}
 \label{4111}
  \neg (( p \vee q ) \vee r )
 \end{equation}

 and 
 
 \begin{equation}
 \label{4112}
  \neg ( \neg p \wedge \neg q )
 \end{equation}


Continueing for Formula \ref{4111} and applying $\alpha$ formula $\neg ( A_1 \vee A_2 )$ we get

\begin{equation}
\label{41111}
 \neg (p \vee q), \neg r 
\end{equation}

Then applying formula $\neg ( A_1 \vee A_2 )$ again we get

\begin{equation}
\label{411111}
 \neg p, \neg q, \neg r
\end{equation}

Which is satisfiable.

Going back to Formula \ref{4112} which can by simplified to 

\begin{equation}
 p \vee q
\end{equation}

Then applying $\beta$ formula $B_1 \vee B_2$ we get formulas

\begin{equation}
 \label{41121}
 p
\end{equation}

and 

\begin{equation}
\label{41122}
q
\end{equation}

Both of which are satifyable. Now looking at all the leaf nodes namely 
Formalas \ref{412}, \ref{411111}, \ref{41121} and \ref{41122} we can see that they are all satisfiable. Thus
Formula \ref{41} is valid.







\section{Question 4.2}

\begin{equation}
\label{420}
(p \rightarrow ( q \rightarrow r )) \leftrightarrow (( p \wedge q) \wedge \neg r) 
\end{equation}

By substituting for double implication operator:

\begin{equation}
 (p \rightarrow ( q \rightarrow r )) \rightarrow (( p \wedge q) \wedge \neg r), (( p \wedge q) \wedge \neg r) \rightarrow (p \rightarrow ( q \rightarrow r )) 
\end{equation}

Substituting for the implication operator in the first term gives:

\begin{equation}
\label{421}
 \neg(p \rightarrow ( q \rightarrow r )),  (( p \wedge q) \wedge \neg r) \rightarrow (p \rightarrow ( q \rightarrow r )) 
\end{equation}

and

\begin{equation}
\label{422}
 ((p \wedge q) \wedge \neg r ),  (( p \wedge q) \wedge \neg r) \rightarrow (p \rightarrow ( q \rightarrow r )) 
\end{equation}

Equation \ref{421} becomes

\begin{equation}
p, \neg ( q \rightarrow r )),  (( p \wedge q) \wedge \neg r) \rightarrow (p \rightarrow ( q \rightarrow r )) 
\end{equation}

then 

\begin{equation}
\label{421a}
p, q, \neg r, (( p \wedge q) \wedge \neg r) \rightarrow (p \rightarrow ( q \rightarrow r )) 
\end{equation}

Substituting for the implication in the above formula gives:

\begin{equation}
\label{4211}
p, q, \neg r, \neg (( p \wedge q) \wedge \neg r)
\end{equation}

and

\begin{equation}
\label{4212}
p, q, \neg r, (p \rightarrow ( q \rightarrow r )) 
\end{equation}


Formula \ref{4211} becomes 


\begin{equation}
\label{42111}
p, q, \neg r, \neg ( p \wedge q)
\end{equation}

and 

\begin{equation}
\label{42112}
p, q, \neg r, r
\end{equation}

We can see that Formula \ref{42112} is unsatifyable.  Then Formula \ref{42111} becomes

\begin{equation}
\label{421111}
p, q, \neg r, \neg p 
\end{equation}

and 

\begin{equation}
\label{421112}
p, q, \neg r, \neg q 
\end{equation}

Both of which are unsatifyable. Now having done all the leaves under Formula \ref{4211}, we continue with the leaves under Equation \ref{4212}.

Equation \ref{4212} becomes

\begin{equation}
\label{42121}
p, q, \neg r, \neg p 
\end{equation}

which is unsatifyable and
 
\begin{equation}
\label{42122}
p, q, \neg r, q \rightarrow r 
\end{equation} 

which becomes

\begin{equation}
\label{421221}
p, q, \neg r, \neg q,  r 
\end{equation} 
 
Having finished all the leaves under Formula \ref{421} we continue with Formula \ref{422} which becomes

\begin{equation}
\label{4221}
 p, q, \neg r,  (( p \wedge q) \wedge \neg r) \rightarrow (p \rightarrow ( q \rightarrow r )) 
\end{equation} 

But Formula \ref{4221} is the same as Formula \ref{421a}. Formula \ref{421a} will have the same leaf nodes as Formula \ref{4221}.

Thus we have all the leaf nodes namely Formulas \ref{42112}, \ref{421111}, \ref{421112}, \ref{42121} and \ref{421221}. All of these Formulas are unsatisfiable, 
therefore the original Formula \ref{420} is unsatisfiable.

\section{Question 4.3}

For formula 

\begin{equation}
\label{430}
(p \rightarrow (q \rightarrow r)) \rightarrow ((p \rightarrow q)  \rightarrow r)
\end{equation}

We have to prove that it is satisfiable and falsifiable using semantic tableaux. Applying the $\beta$ formula fot implication we get

\begin{equation}
\label{431}
 \neg (p \rightarrow (q \rightarrow r)) 
\end{equation}

and

\begin{equation}
 \label{432}
 ((p \rightarrow q)  \rightarrow r)
\end{equation}

Applying the $\alpha$ formula for negated implication, set \ref{431} becomes


\begin{equation}
\label{4311}
p, \neg (q \rightarrow r)) 
\end{equation}

Applying the $alpha$ formula for negated implication to set \ref{4311} we get

\begin{equation}
\label{4311}
p,q,\neg r
\end{equation}

We have found a set that does not contain an atom and its negation. Proving formula \ref{430} is satisfiable.


\section{Question 5}





\end{document}
