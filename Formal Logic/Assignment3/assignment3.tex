\documentclass[10pt,a4paper]{article}
\usepackage[natbibapa]{apacite} 
\usepackage{amsmath}
\usepackage{textcomp}
\usepackage[a4paper,margin=0.8in]{geometry} 
\usepackage{latexsym}
\usepackage{tikz}
\usepackage{etoolbox}
\usepackage{mathrsfs}

%opening
\title{COS 4807 Assignment 3}
\author{Adriaan Louw (53031377)}

\newcommand{\interp}{\mathscr{I}}


\tikzset{
  treenode/.style = {shape=rectangle, rounded corners,
                     draw, align=center,
                     top color=white, bottom color=blue!20},
  root/.style     = {treenode, font=\Large, bottom color=red!30},
  env/.style      = {treenode, font=\ttfamily\normalsize},
  dummy/.style    = {circle,draw}
}

\usetikzlibrary{shapes.misc}


\begin{document}

\maketitle


\section{Question 1i}
Let $\interp$ be an arbitrary interpretation such that $v_\interp(\forall x p(x)\vee\exists x q(x))=F$. From the truth value of disjunction $v_\interp(\forall xp(x) = F $ and $v_\interp\exists x q(x)) = F$. From this using Theorem 7.22 we for all assignments $v_{\sigma\interp}(p(x)) = F$ and for some assignments $v_{\sigma\interp}(q(x)) = F$. Then by the truth value of disjunction $v_{\sigma\interp}(p(x)\vee q(x)) = F$. Then by using Theorem 7.22 $v_I(\forall(p(x)\vee q(x))) = F$. Then if $v_I(\forall x p(x) \vee \exists x q(x)) = F$ then $v_\interp (\forall x(p(x)\vee q(x) \rightarrow(\forall x p(x)\vee \exists x q(x)) = T))$ by the truth value of implication. And since $\interp$ is an arbitrary interpretation, the formula is valid

\section{Question 1ii}
Let $\interp$ ba an arbitrary interpretation such that $v_\interp (\forall x \neg p(x)\vee\forall x \neg q(x))= F$. Then from the definition of disjunction $v_\interp (\forall x \neg p(x)) = F$ and $v_\interp (\forall x\neg q(x))=F$. Using the theorem from question 3ii we get for all assignments $v_{\sigma\interp} (\neg p(x))=F$ and for all assignmets $v_{\sigma\interp}(\neg q(x))=F$. Then by the truth values of negation, $v_{\sigma\interp}p(x) =T$ and $v_{\sigma\interp}q(x)= T$. Then by theorem 7.22 and the definition of conjunction $v_\interp (\exists x p(x)\wedge q(x))=T$. Now we have shown that  $v_\interp (\forall x \neg p(x)\vee\forall x \neg q(x))= F$ and $v_\interp \exists (x p(x)\wedge q(x))=T$. Combining these into the original formula we get $v_\interp( \exists x (p(x)\wedge q(x)) \wedge (\forall x \neg p(x)\vee\forall x \neg q(x)))= F$ by the definition of conjunction. And since $\interp$ is an arbitrary interpretation, the formula is unsatifiable.








\end{document}
