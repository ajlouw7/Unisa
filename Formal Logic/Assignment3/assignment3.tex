\documentclass[10pt,a4paper]{article}
\usepackage[natbibapa]{apacite} 
\usepackage{amsmath}
\usepackage{textcomp}
\usepackage[a4paper,margin=0.8in]{geometry} 
\usepackage{latexsym}
\usepackage{tikz}
\usepackage{etoolbox}
\usepackage{mathrsfs}

%opening
\title{COS 4807 Assignment 3}
\author{Adriaan Louw (53031377)}

\newcommand{\interp}{\mathscr{I}}


\tikzset{
  treenode/.style = {shape=rectangle, rounded corners,
                     draw, align=center,
                     top color=white, bottom color=blue!20},
  root/.style     = {treenode, font=\Large, bottom color=red!30},
  env/.style      = {treenode, font=\ttfamily\normalsize},
  dummy/.style    = {circle,draw}
}

\usetikzlibrary{shapes.misc}


\begin{document}

\maketitle


\section{Question 1i}
Let $\interp$ be an arbitrary interpretation such that $v_\interp(\forall x p(x)\vee\exists x q(x))=F$. From the truth value of disjunction $v_\interp(\forall xp(x) = F $ and $v_\interp\exists x q(x)) = F$. From this using Theorem 7.22 we for all assignments $v_{\sigma\interp}(p(x)) = F$ and for some assignments $v_{\sigma\interp}(q(x)) = F$. Then by the truth value of disjunction $v_{\sigma\interp}(p(x)\vee q(x)) = F$. Then by using Theorem 7.22 $v_I(\forall(p(x)\vee q(x))) = F$. Then if $v_I(\forall x p(x) \vee \exists x q(x)) = F$ then $v_\interp (\forall x(p(x)\vee q(x) \rightarrow(\forall x p(x)\vee \exists x q(x)) = T))$ by the truth value of implication. And since $\interp$ is an arbitrary interpretation, the formula is valid

\section{Question 1ii}
Let $\interp$ ba an arbitrary interpretation such that $v_\interp (\forall x \neg p(x)\vee\forall x \neg q(x))= F$. Then from the definition of disjunction $v_\interp (\forall x \neg p(x)) = F$ and $v_\interp (\forall x\neg q(x))=F$. Using the theorem from question 3ii we get for all assignments $v_{\sigma\interp} (\neg p(x))=F$ and for all assignmets $v_{\sigma\interp}(\neg q(x))=F$. Then by the truth values of negation, $v_{\sigma\interp}p(x) =T$ and $v_{\sigma\interp}q(x)= T$. Then by theorem 7.22 and the definition of conjunction $v_\interp (\exists x p(x)\wedge q(x))=T$. Now we have shown that  $v_\interp (\forall x \neg p(x)\vee\forall x \neg q(x))= F$ and $v_\interp \exists (x p(x)\wedge q(x))=T$. Combining these into the original formula we get $v_\interp( \exists x (p(x)\wedge q(x)) \wedge (\forall x \neg p(x)\vee\forall x \neg q(x)))= F$ by the definition of conjunction. And since $\interp$ is an arbitrary interpretation, the formula is unsatifiable.

\section{Question 2i}

\begin{equation}
 \forall x(\neg p(x) \leftrightarrow \exists y(q(a,x,y) \wedge r(x,y))
\end{equation}

\section{Question 2ii}
$\{\mathscr{N},prime,equals,less\_than\}$


\section{Question 3i}



\section{Question 3ii}


\section{Question 4i}

To prove that 
\begin{equation}
 \forall x(p(x)\vee q(x)) \rightarrow (\forall x p(x)\vee\exists x q(x))
\end{equation}

is valid with sematic tableau we need to show that 

\begin{equation}
 \neg \forall x(p(x)\vee q(x)) \rightarrow (\forall x p(x)\vee\exists x q(x))
\end{equation}

closes. 

Applying rule $\neg(A_1\rightarrow A_2)$ we get

\begin{equation}
 \forall x(p(x)\vee q(x)),\neg(\forall x p(x)\vee\exists x q(x)))
\end{equation}

Applying rule $\neg(A_1 \vee A_2)$ we get

\begin{equation}
 \forall x(p(x)\vee q(x)),\neg\forall xp(x),\neg\exists xq(x)
\end{equation}

Using the dual of the 2 quantifiers
\
\begin{equation}
 \forall x(p(x)\vee q(x)),\exists\neg xp(x),\forall\neg xq(x)
\end{equation}

Using the $\gamma$ rule ofr existential qualification

\begin{equation}
 \forall x(p(x)\vee q(x)),\neg p(a_1),\forall\neg xq(x)
\end{equation}

\begin{equation}
 \forall x(p(x)\vee q(x)),p(a_1)\vee q(a_1),\neg p(a_1),\forall\neg xq(x),\neg q(a_1)
 \end{equation}
 
 Using $B_1\vee B_2$the branches split into
 
 \begin{equation}
 \forall x(p(x)\vee q(x)),p(a_1),\neg p(a_1),\forall\neg xq(x),\neg q(a_1)
 \end{equation}
 
 and
 
 \begin{equation}
 \forall x(p(x)\vee q(x)), q(a_1),\neg p(a_1),\forall\neg xq(x),\neg q(a_1)
 \end{equation}
 
 Both of which close. THus the original formula is valid

 
 
 \section{Question 4ii}
 
 
 We need to prove that
 
 \begin{equation}
  \exists x(p(x)\wedge q(x))\wedge(\forall x\neg p(x) \vee \forall x \neg q(x))
 \end{equation}

 is unsatisfiable. Thus the tableau for this formula has to close.
 
 Applying rule $A_1\wedge A_2$
 
 \begin{equation}
  \exists x(p(x)\wedge q(x)),\forall x\neg p(x) \vee \forall x \neg q(x)
 \end{equation}
 
 Applying rule $B_1 \vee B_2$ we get 
 
 \begin{equation}
 \label{421}
 \exists x(p(x)\wedge q(x)),\forall x \neg p(x)
 \end{equation}

 \begin{equation}
 \label{422}
  \exists x(p(x)\wedge q(x)),\forall x \neg q(x)
 \end{equation}
 
 For branch of Equation \ref{421}
 
 \begin{equation}
  p(a_1) \wedge q(a_1),\forall x\neg p(x)
 \end{equation}
 
  \begin{equation}
  p(a_1) \wedge q(a_1),\forall x\neg p(x),\neg p(a_1)
 \end{equation}
 
using $A_1\wedge A_2$

  \begin{equation}
  p(a_1), q(a_1),\forall x\neg p(x),\neg p(a_1)
 \end{equation}
 
 which closes.
 
 Now for the branch of Equation \ref{422}

 \begin{equation}
  p(a_1) \wedge q(a_1),\forall x\neg q(x)
 \end{equation}
 
  \begin{equation}
  p(a_1) \wedge q(a_1),\forall x\neg q(x),\neg q(a_1)
 \end{equation}
 
using $A_1\wedge A_2$

  \begin{equation}
  p(a_1), q(a_1),\forall x\neg q(x),\neg q(a_1)
 \end{equation}
 
 which closes.

 Thus the original formula is unsatisfiable
 
 
 \section{Question 5i}
 We need to show that the tableau for this formula has an open branch
 \begin{equation}
 (\forall x(p(x) \vee q(x)))\rightarrow (\forall x p(x) \vee \forall xq(x))
\end{equation}

Using rule $B_1 \rightarrow B_2$ we get

\begin{equation}
\label{511}
 \neg\forall x (p(x)\vee q(x))
\end{equation}

and 

\begin{equation}
\label{512}
 \forall x p(x) \vee \forall x q(x)
\end{equation}

We will continue with Equation \ref{511}

\begin{equation}
 \exists \neg x (p(x)\vee q(x))
\end{equation}

\begin{equation}
 \neg(p(a_1)\vee q(a_1))
\end{equation}

Using rule $\neg(A_1\vee A_2)$

\begin{equation}
 \neg p(a_1),\neg q(a_1)
\end{equation}

Which is an open branch. So the original formula is satisfiable and there is no need to continue with the other branch on Equation \ref{512} 










 \section{Question 5ii}
 
 We need to prove that the negation of the formula is satisfiable i.e. has an open branch
 
 
\begin{equation}
 \neg ((\forall x(p(x) \vee q(x)))\rightarrow (\forall x p(x) \vee \forall xq(x)))
\end{equation}

Using rule $\neg(A_1 \rightarrow A_2)$

\begin{equation}
 \forall x (p(x) \vee q(x)),\neg(\forall x p(x) \vee \forall x q(x))
\end{equation}

 Using rule $\neg(A_1 \vee A_2)$

 \begin{equation}
 \forall x (p(x) \vee q(x)), \neg \forall x p(x), \neg \forall x q(x)
\end{equation}

 \begin{equation}
 \forall x (p(x) \vee q(x)), \exists\neg x p(x), \exists \neg x q(x)
\end{equation}

 \begin{equation}
 \forall x (p(x) \vee q(x)),\neg p(a_1), \exists \neg x q(x)
\end{equation}

 \begin{equation}
 \forall x (p(x) \vee q(x)),p(a_1)\vee q(a_1) ,\neg p(a_1), \exists \neg x q(x)
\end{equation}
 \begin{equation}
 \forall x (p(x) \vee q(x)),p(a_1)\vee q(a_1) ,\neg p(a_1),  \neg q(a_2)
\end{equation}

 \begin{equation}
 \forall x (p(x) \vee q(x)),p(a_1)\vee q(a_1) ,\neg p(a_1),  \neg q(a_2), p(a_2)\vee q(a_2)
\end{equation}

 Using rule $B_1 \vee B_2$ on the term $p(a_1)\vee q(a_1)$ we get 2 branches.
 
  \begin{equation}
  \label{51}
 \forall x (p(x) \vee q(x)),p(a_1) ,\neg p(a_1),  \neg q(a_2), p(a_2)\vee q(a_2)
\end{equation}

and 

 \begin{equation}
 \label{52}
 \forall x (p(x) \vee q(x)), q(a_1) ,\neg p(a_1),  \neg q(a_2), p(a_2)\vee q(a_2)
\end{equation}
 
Equation \ref{51} closes so we continue with Equation \ref{52}. Using rule $B_1 \vee B_2$ gives 

 \begin{equation}
 \label{53}
 \forall x (p(x) \vee q(x)), q(a_1) ,\neg p(a_1),  \neg q(a_2), p(a_2)
\end{equation} 

and

\begin{equation}
 \label{54}
 \forall x (p(x) \vee q(x)), q(a_1) ,\neg p(a_1),  \neg q(a_2), q(a_2)
\end{equation}

Equation \ref{54} closes but Equation \ref{53} is an open branch. Therefor the original equation is falsifiable.

 
\end{document}
